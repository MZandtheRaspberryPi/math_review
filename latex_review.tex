

\documentclass{article}
\usepackage{amssymb,amsmath}
% useful: http://tug.ctan.org/info/short-math-guide/short-math-guide.pdf
\begin{document}
This is an inline formula $ 3 + 3 $

So is this: \(3+3\)

This is a displayed equation, unnumbered.
\[3+3\]

So is this:
\begin{equation*}
3+3
\end{equation*}


This is a displayed equation, numbered.
\begin{equation}
3+3
\end{equation}

So is this
\begin{equation}
3+3
\end{equation}

\begin{equation}\label{xx} % label here, if we want to reference this equation by some other tag. \eqref to ref manually tagged equations. , \eqref{ax1}
\begin{split}
a& =b+c-d\\
& \quad +e-f\\
& =g+h\\
& =i
\end{split}
\end{equation}

\begin{multline}
a+b+c+d+e+f\\
+i+j+k+l+m+n\\
+o+p+q+r+s
\end{multline}

\begin{gather}
a_1=b_1+c_1\\
a_2=b_2+c_2-d_2+e_2
\end{gather}

\begin{align}
a_1& =b_1+c_1\\
a_2& =b_2+c_2-d_2+e_2
\end{align}

\begin{align}
a_{11}& =b_{11}&
a_{12}& =b_{12}\\
a_{21}& =b_{21}&
a_{22}& =b_{22}+c_{22}
\end{align}

\begin{alignat}{2}
a_1& =b_1+c_1& &+e_1-f_1\\
a_2& =b_2+c_2&{}-d_2&+e_2
\end{alignat}

\begin{flalign} %full length, occupying full width often begins at left margin
a_{11}& =b_{11}&
a_{12}& =b_{12}\\
a_{21}& =b_{21}&
a_{22}& =b_{22}+c_{22}
\end{flalign}

% * to any primary environment will supress equation numbering.


\begin{equation}\label{first}
a=b+c
\end{equation}
some intervening text
\begin{subequations}\label{grp} % sub equations here to do subnumbering
\begin{align}
a&=b+c\label{second}\\
d&=e+f+g\label{third}\\
h&=i+j\label{fourth}
\end{align}
\end{subequations}


% Matrices:
\begin{equation}
\begin{pmatrix}
\alpha& \beta^{*}\\
\gamma^{*}& \delta
\end{pmatrix}
\end{equation}

% fractions
\begin{equation}
\frac{1}{k}\log_2 c(f),\quad\dfrac{1}{k}\log_2 c(f),
\quad\tfrac{1}{k}\log_2 c(f)
\end{equation}

%binomial
\begin{equation}
2^k-\binom{k}{1}2^{k-1}+\binom{k}{2}2^{k-2}
\end{equation}

\end{document}